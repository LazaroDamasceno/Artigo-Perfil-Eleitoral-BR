\section{Procedimentos metodológicos}

A figura \ref{fig:projecaopopulacaobr} mostra como a população do brasileira crescerá e declinará entre 2000 e 2070. O  período de crescimento engloba 2000 até 2041; o declínio, 2042 até 2070. A população atinge o pico do número de habitantes em 2041 com 220.425.299 habitantes. 

O ano em que o Brasil menos teve habitantes foi 2000, com 174.695.935 pessoas. Comparativamente, 2013 é o ano cuja população mais se aproxima da população de 2070. 2013 tem 199.226.702 habitantes; já 2070, 199.228.708, ou seja, a diferença é de apenas 2.006 pessoas. 

A linha vermelha indica o ano em que haverá o pico populacional. A inclusão da linha vermelha pontilhada na figura \ref{fig:projecaopopulacaobr} é a  marca visual do ponto de ruptura na crescimento da população para seu declínio previsto, pois após o ano seguinte ao pico do número de habitantes, a população brasileiro começa a declinar.

O período de crescimento populacional representa 59,15\%, ou seja, 42 anos de crescimento constante da população, mesmo no período da pandemia do Covid-19, enquanto o período de declínio representa 40,85\%, ou seja, 29 anos de perda de habitantes prevista.

A mudança demográfica representada na figura \ref{fig:projecaopopulacaobr} mostra como a população muda ao longo das décadas. No entanto, como o ano presente ainda não é 2041 ou 2042, outro problema é o mais calamitante: o envelhecimento populacional.