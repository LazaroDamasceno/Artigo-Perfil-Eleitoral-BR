\section{Procedimentos metodológicos}

A figura \ref{fig:projecaopopulacaobr} mostra como a população do brasileira crescerá e declinará entre 2000 e 2070. O  período de crescimento engloba 2000 até 2041; o declínio, 2042 até 2070. A população atinge o pico do número de habitantes em 2041 com 220.425.299 habitantes. 

O ano em que o Brasil menos teve habitantes foi 2000, com 174.695.935 pessoas. Comparativamente, 2013 é o ano cuja população mais se aproxima da população de 2070. 2013 tem 199.226.702 habitantes; já 2070, 199.228.708, ou seja, a diferença é de apenas 2.006 pessoas. 

A linha vermelha indica o ano em que haverá o pico populacional. A inclusão da linha vermelha pontilhada na figura \ref{fig:projecaopopulacaobr} é a  marca visual do ponto de ruptura na crescimento da população para seu declínio previsto, pois após o ano seguinte ao pico do número de habitantes, a população brasileiro começa a declinar.

O período de crescimento populacional representa 59,15\%, ou seja, 42 anos de crescimento constante da população, mesmo no período da pandemia do Covid-19, enquanto o período de declínio representa 40,85\%, ou seja, 29 anos de perda de habitantes prevista.

A mudança demográfica representada na figura \ref{fig:projecaopopulacaobr} mostra como a população muda ao longo das décadas. No entanto, como o ano presente ainda não é 2041 ou 2042, outro problema é o mais calamitante: o envelhecimento populacional.

Com base nos mesmos dados usados para gerar a figura \ref{fig:projecaopopulacaobr}, foi criada a figura \ref{fig:projecaojovensidosos}. Para a transformação de dados que geraram esta figura, foi necessário entender o conjunto de dados.

No conjunto de dados da projeção da população que agrupa as pessoas por grupo etários quinquenais, há as variáveis grupo etário, sexo, código, sigla, local e os anos de 2000 até 2070. Na variável sexo, há os valores ambos, masculino e feminino. Nas variáveis local siglas, respectivamente , há os valores Brasil, as unidades federativas e as grandes regiões e suas siglas.

Das variáveis local e sexo foram escolhidos os valores Brasil e ambos, respectivamente. O próximo passo foi a remoção das variáveis sexo, local, sigla e código, deixando apenas das variáveis grupo etário e os anos.

Os grupos etários escolhidos se enquadram na definição legal de jovem e idoso. De acordo com \textcite{estatuto_juventude}, a idade da pessoa jovem varia entre 15 e 29 anos; já \textcite{estatuto_idoso} define idoso como a pessoas que tem 60 ou mais.

Nesse sentido, foram escolhidos os grupos etários 15-19, 20-24, 25-29 para os jovens e os grupos etários 60-64, 65-69, 70-74, 75-79, 80-84, 85-89 e 90+ para os idosos. Para cada grupo, as colunas dos anos tiveram seus valores somados para descobrir qual era, é ou será a população da parcela selecionada.

Considerando a contextualização do tratamento de dados, que resultou na figura \ref{fig:projecaojovensidosos}, ela mostra como a população jovem cresceu de 2000 até 2009 e começou a cair desde 2010. Diferentemente da população jovem, a idosa tem crescido sem interrupção. 

É no ano de 2032 que a população jovem estará apenas um pouco menor do que idosa. A linha vermelha pontilhada marca o ano 2032 como o ponto de convergência. O pico de número de pessoas jovem foi em ?, quando se tinha ? de jovens. Em 2032, a diferença entre a população jovem da idosa será de ?. Em 2070, haverá ? de jovens e ? de idosos.