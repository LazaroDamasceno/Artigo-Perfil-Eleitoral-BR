\section{Perfilamento dos eleitores}

No Apêndice \ref{figuras_grandes} está presente duas figuras. A figura \ref{fig:projecaopopulacaoano} mostra o crescimento e declínio da população entre 2000  e 2070. Complementarmente, a figura \ref{fig:projecaopopulacaogrupoetarioano} detalha a figura anterior, focando nas populações jovens (15-29 anos) e idosas (60+ anos).

Ambas as figuras são 14 polegadas (largura) e 10 polegada (altura) para garantir clareza visual motivada pelo período de 71 anos no eixo X. Uma polegada equivale a 2,54 cm. 

Para evitar fadiga visual, usou a paleta de cores Dark2 e o cor azul padrão do matplotlib. Será mantida a notação científica para os valores numéricos do eixo Y, pois seu uso permite que as figuras tenham proporções grandes.

Da figura \ref{fig:projecaopopulacaoano}, nota-se como a população mostra tem crescido cresceu ininterruptivelmente de 2000 até 2041 e declinará de 2042 de 2070. A linha pontilhada vermelha em 2041 marca o pico populacional quando o Brasil terá 220.425.299 habitantes. No ano 2000, o país tinha 174.695.935 habitantes. 

O número de habitantes em 2070 será 199.228.708 habitantes. Comparavelmente, a população de 2070 será muito similar a de 2013, quando o país tinha 199.226.702 habitantes, ou seja, uma diferença positiva de apenas 2.006 pessoas para 2070.

Para compreender os grupos etários da figura \ref{fig:projecaopopulacaogrupoetarioano}, deve-se considerar dois aspectos fundamentais: quais são as definições legais de jovem e idoso e como Instituto Brasileiro de Geografia e Estatística (IBGE) divide a população na projeção.

De acordo com o Estatuto da Juventude, conforme expressa \textcite{estatuto_jovens}, considera-se legalmente jovem quando a idade do cidadão varia entre 15 e 29 anos. Já para o Estatuto da Pessoa Idosa, segundo \textcite{estatuto_idoso}, considera-se como pessoas idosa aquela que tem pelos menos 60 anos ou mais.

Além disso, é importante informar que o IBGE dividiu a população por idade ou quinquenalmente. Optou-se por este. Nessa divisão, as idades são agrupadas a cada 5 anos, porém há uma exceção: o grupo etário 90+. Para fins de análise exploratória de dados, manteve-se a forma como o IBGE dividiu a população jovem, ou sejam, em grupos de idade (15-19, 20-24, 25-29).

Para a população idosa, cujos grupos etários são 60-64, 65-69, 70-74, 75-79, 80-84, 85-89 e 90+. Com exceção do grupo 90+, todos foram reagrupados decimalmente, assim 7 grupos foram reduzidos a 4 (60-69, 70-79, 80-89, 90+). A mudança foi motivada pela clareza visual esperada para a figura \ref{fig:projecaopopulacaogrupoetarioano}.

Detalhadamente, a figura \ref{fig:projecaopopulacaogrupoetarioano} mostra com a população jovem e idosa estão indo para direções contrárias: a jovem declina; a idosa aumenta. O crescimento da população idosa é contínuo, sendo a que mais crescerá será a população idosa cuja grupo etário é 60-69 anos. A que menos crescerá será a população idosa cujo grupo etário é o de 90+.