\section{Perfilamento dos eleitores}

No Apêndice \ref{figuras_grandes} está presente duas figuras. A figura \ref{fig:projecaopopulacaoano} mostra o crescimento e declínio da população entre 2000  e 2070. Complementarmente, a figura \ref{fig:projecaopopulacaogrupoetarioano} detalha a figura anterior, focando nas populações jovens (15-29 anos) e idosas (60+ anos). Ambas as figuras são 14 polegadas (largura) e 10 polegada (altura). Uma polegada equivale a 2,54 cm.

Da figura \ref{fig:projecaopopulacaoano}, nota-se como a população mostra tem crescido cresceu ininterruptivelmente de 2000 até 2041 e declinará de 2042 de 2070. A linha pontilhada vermelha em 2041 marca o pico populacional quando o Brasil terá 220.425.299 habitantes. No ano 2000, o país tinha 174.695.935 habitantes. 

O número de habitantes em 2070 será 199.228.708 habitantes. Comparavelmente, a população de 2070 será muito similar a de 2013, quando o país tinha 199.226.702 habitantes, ou seja, uma diferença positiva de apenas 2.006 pessoas para 2070.

?

Detalhadamente, a figura \ref{fig:projecaopopulacaogrupoetarioano} mostra com a população jovem e idosa estão indo para direções contrárias: a jovem declina; a idosa aumenta. O crescimento da população idosa é contínuo, sendo a que mais crescerá será a população idosa cuja grupo etário é 60-69 anos. A que menos crescerá será a população idosa cujo grupo etário é o de 90+.